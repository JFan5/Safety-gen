\section{Related Work}\label{sec:related}
This section introduces the related work on classical task planning, AI-based planners, and LLM-based planning.

\subsection{Classical Task Planning}

Classical task planning is a mature area in automated reasoning, with planners such as
Fast Downward \cite{helmert2006fast}, Metric-FF \cite{hoffmann2001ff}, and OPTIC \cite{benton2012temporal},
solving symbolic planning problems by heuristic search over PDDL models.
Extensions including Hierarchical Task Networks (HTN) \cite{nau2005applications} and other model-based planners \cite{geffner2013concise} improve scalability through structural decomposition and domain-specific abstractions.

Despite their success, classical planners exhibit several well-recognized limitations. They rely heavily on expert-designed domain models and heuristic functions \cite{geffner2013concise}, making the modeling pipeline labor-intensive and brittle to environmental variations \cite{ghallab2004automated}. Moreover, as search-based methods, their computational cost often scales poorly with problem size, leading to rapidly increasing planning time as task complexity grows \cite{benton2012temporal}. These constraints motivate the shift toward more flexible,
data-driven planning paradigms.


\subsection{Learning-based Planners}
To address the rigidity of classical planners, a broad line of learning-driven
approaches has emerged. Early work focuses on learning heuristics or search guidance
from data \cite{yoon2008learning,shivashankar2015hierarchical}, enabling planners
to approximate expert reasoning without relying on manually crafted heuristics.
RL-based methods have also been explored to derive task policies from
environment interaction. However, RL-based planners often suffer from poor scalability due
to state and action space explosion, require extensive rollout-based data collection, and
exhibit limited generalization to novel goals or task configurations~ \cite{shivadekar2025artificial}. Neuro-symbolic \cite{acharya2023neurosymbolic} and hybrid approaches \cite{lyu2022towards} have explored integrating symbolic reasoning with learned components , but often remain domain-specific and require substantial manual engineering.

Overall, despite substantial progress, existing learning-based planners do not achieve
the joint requirements of (i) strong generalization across tasks and domains and
(ii) verifiable adherence to formal safety constraints—two properties that are essential
for robotic deployment.


\subsection{LLM-based Planning}
Recent work has investigated LLMs as planners that directly
generate or reason over plans for robotic and task planning.
For embodied agents, LLMs have demonstrated strong few-shot planning capabilities by
generating and updating plans grounded in environmental observations~\cite{song2023llm}.
LLMs have also been explored as generalized planners that synthesize domain-level planning
programs from a small number of training tasks, enabling efficient plan generation and
within-domain generalization~\cite{silver2024generalized}. To improve plan feasibility and correctness, several methods incorporate external
validation or refinement mechanisms.
ISR-LLM adopts an iterative self-refinement process with validation feedback to enhance
long-horizon planning performance~\cite{zhou2024isr}, while hybrid approaches combine LLM-based
reasoning with classical heuristic search planners for long-horizon or multi-agent
planning~\cite{zhang2025lamma}.
Recent benchmark studies further indicate that LLM planning performance can be
overestimated under simplistic settings, as introducing fine-grained constraints exposes
robustness and safety limitations~\cite{huang2025language}.


Overall, although recent LLM-based planners show promising planning performance, most approaches rely on frozen or lightly adapted models and external validation, and do not intrinsically learn safety-aware planning behavior or systematically generalize safety constraints across tasks and domains.

