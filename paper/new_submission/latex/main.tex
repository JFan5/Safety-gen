%%
%% This is file `sample-acmcp.tex',
%% generated with the docstrip utility.
%%
%% The original source files were:
%%
%% samples.dtx  (with options: `all,journal,acmcp')
%% 
%% IMPORTANT NOTICE:
%% 
%% For the copyright see the source file.
%% 
%% Any modified versions of this file must be renamed
%% with new filenames distinct from sample-acmcp.tex.
%% 
%% For distribution of the original source see the terms
%% for copying and modification in the file samples.dtx.
%% 
%% This generated file may be distributed as long as the
%% original source files, as listed above, are part of the
%% same distribution. (The sources need not necessarily be
%% in the same archive or directory.)
%%
%%
%% Commands for TeXCount
%TC:macro \cite [option:text,text]
%TC:macro \citep [option:text,text]
%TC:macro \citet [option:text,text]
%TC:envir table 0 1
%TC:envir table* 0 1
%TC:envir tabular [ignore] word
%TC:envir displaymath 0 word
%TC:envir math 0 word
%TC:envir comment 0 0
%%
%% The first command in your LaTeX source must be the \documentclass
%% command.
%%
%% For submission and review of your manuscript please change the
%% command to \documentclass[manuscript, screen, review]{acmart}.
%%
%% When submitting camera ready or to TAPS, please change the command
% to \documentclass[sigconf]{acmart} or whichever template is required
%% for your publication.
%%
%%

% \documentclass[sigconf, screen, review]{acmart}
\documentclass[sigconf,review]{acmart}
\usepackage{algorithmic}
\usepackage{tcolorbox}
\usepackage{tabularx} 
\usepackage{xcolor}
\usepackage{listings}
\usepackage{subcaption}



\tcbuselibrary{listings, breakable}

\newtcolorbox{userbox}{
  colback=gray!3!white,   
  colframe=gray!40!black,  
  title=Prompt Template,
  fonttitle=\bfseries,
  breakable
}


\lstdefinelanguage{PDDL}{
  morekeywords={define,domain,requirements,types,predicates,action,parameters,
                precondition,effect,init,goal,problem,when,not,and,or},
  sensitive=false
}

\lstdefinestyle{lispstyle}{
  language=PDDL,
  basicstyle=\ttfamily\small,       % 字体样式
  keywordstyle=\bfseries\color{blue!60!black},
  commentstyle=\itshape\color{gray!80!black},
  columns=fullflexible,
  keepspaces=true,
  showstringspaces=false,
  frame=single,                     % 单线边框
  rulecolor=\color{black!20},
  backgroundcolor=\color{gray!3},
  frameround=tttt,                  % 圆角边框
  breaklines=true,                  % 自动换行
  captionpos=b,                     % 标题在下方
  xleftmargin=1em, xrightmargin=1em
}


\settopmatter{printacmref=false} 
\setcopyright{none}              


%% \BibTeX command to typeset BibTeX logo in the docs
\AtBeginDocument{%
  \providecommand\BibTeX{{%
    Bib\TeX}}}

%% Rights management information.  This information is sent to you
%% when you complete the rights form.  These commands have SAMPLE
%% values in them; it is your responsibility as an author to replace
%% the commands and values with those provided to you when you
%% complete the rights form.
% \setcopyright{acmlicensed}
% \copyrightyear{2018}
% \acmYear{2018}
% \acmDOI{XXXXXXX.XXXXXXX}

%%
%% These commands are for a JOURNAL article.
% \acmJournal{JDS}
% \acmVolume{37}
% \acmNumber{4}
% \acmArticle{111}
% \acmMonth{8}

%%
%% Submission ID.
%% Use this when submitting an article to a sponsored event. You'll
%% receive a unique submission ID from the organizers
%% of the event, and this ID should be used as the parameter to this command.
%%\acmSubmissionID{123-A56-BU3}

\newcommand{\weizhe}[1]{\textcolor{blue}{[Weizhe: #1]}}
\newcommand{\OS}[1]{\textcolor{red}{[OS: #1]}}


%%
%% end of the preamble, start of the body of the document source.
\begin{document}


\title{SafeGen-LLM: Enhancing Safety Generalization in Task Planning for Cyber-Physical Systems}

%%
%% By default, the full list of authors will be used in the page
%% headers. Often, this list is too long, and will overlap
%% other information printed in the page headers. This command allows
%% the author to define a more concise list
%% of authors' names for this purpose.
\renewcommand{\shortauthors}{Trovato et al.}
%%
%% Article type: Research, Review, Discussion, Invited or position
\acmArticleType{Review}
%%
%% Links to code and data
\acmCodeLink{https://github.com/borisveytsman/acmart}
\acmDataLink{htps://zenodo.org/link}
%%
%% Authors' contribution
\acmContributions{BT and GKMT designed the study; LT, VB, and AP
  conducted the experiments, BR, HC, CP and JS analyzed the results,
  JPK developed analytical predictions, all authors participated in
  writing the manuscript.}
%%
%% Sometimes the addresses are too long to fit on the page.  In this
%% case uncomment the lines below and fill them accodingly.
%%
%% \authorsaddresses{Corresponding author: Ben Trovato,
%% \href{mailto:trovato@corporation.com}{trovato@corporation.com};
%% Institute for Clarity in Documentation, P.O. Box 1212, Dublin,
%% Ohio, USA, 43017-6221}

%%
%%
%% Keywords. The author(s) should pick words that accurately describe
%% the work being presented. Separate the keywords with commas.
\begin{abstract}
    Safety-critical task planning in robotic systems remains challenging: classical planners suffer from poor scalability, reinforcement learning (RL)-based methods generalize poorly, and base large language models (LLMs) cannot guarantee safety. We construct a multi-domain Planning Domain Definition Language 3 (PDDL3) benchmark with explicit safety constraints and introduce a two-stage post-training framework: Supervised Fine-Tuning (SFT) on a constraint-compliant planning dataset to learn planning syntax and semantics, followed by Group Relative Policy Optimization (GRPO) guided by fine-grained reward machines derived from formal verification to enforce safety alignment. With curriculum learning for training stability, our approach achieves strong safety generalization across multi-domain planning tasks and outperforms frontier proprietary baselines.
\end{abstract}



\maketitle


\section{Introduction}

Robotic systems tightly integrate computation, communication, and physical processes, and are widely deployed in safety-critical domains such as autonomous driving, industrial automation, and warehouse logistics. Unlike conventional computing systems, robotic systems interact directly with the physical world, where unsafe decisions may lead to irreversible consequences. For instance, in autonomous driving, a planning error can result in collisions; in industrial automation, unsafe operations may damage equipment or harm workers. These examples highlight that robotic task planning must go beyond efficiency and task completion: it must ensure \emph{verifiable safety} under diverse and dynamic operating conditions.

Task planning is a core capability for robotic systems, endowing agents with the ability to organize and execute long-horizon tasks in constrained environments. Traditional task planners are predominantly search-based and operate on formal models expressed in the Planning Domain Definition Language (PDDL). Planners such as Fast Downward~\cite{helmert2006fast} and Metric-FF~\cite{hoffmann2001ff} employ heuristic search over symbolic state spaces to generate plans that can be verified against the underlying model.

% classcial plalner's limitation
However, classical planners exhibit fundamental limitations: (i) \emph{poor scalability}---solving time grows exponentially as problem complexity increases, and (ii) \emph{rigid input/output formats}---requiring substantial domain engineering and hand-crafted heuristics~\cite{geffner2013concise,ghallab2004automated}. When additional factors such as resource limits or safety constraints are introduced, these heuristics and search operators may no longer capture the relevant problem dynamics, leading to degraded performance or infeasible plans.


% RL planner's limitations
Learning-based planners~\cite{wang2022ensuring,yu2021learning} attempt to alleviate these issues by using deep learning or reinforcement learning (RL) to learn heuristics or policies directly from data. Such methods can, in principle, incorporate safety constraints into the learning objective and generate safety-aware policies. However, RL-based planners also face critical limitations: (i) \emph{limited generalization}---trained policies typically handle only a single planning task \cite{packer2018assessing,kansky2017schema}, and (ii) \emph{high data and interaction cost}—achieving reliable performance often requires extensive environment interactions and large numbers of rollouts for training and evaluation~\cite{shivadekar2025artificial, dulac2020empirical}. As a result, their applicability in mission-critical robotic systems is fundamentally constrained.

Recently, large language models (LLMs) have emerged as powerful general-purpose reasoning engines capable of capturing knowledge, following instructions, and generalizing across domains~\cite{cao2025large,plaat2024reasoning,liang2025ai}. LLMs offer high potential for robotic task planning because pretrained models can handle diverse and flexible input formats, from natural-language descriptions to symbolic PDDL specifications. Early studies show that LLMs can generate plausible plans from natural-language or symbolic inputs~\cite{yang2022automaton}, translate instructions into temporal logic specifications~\cite{pan2023data,van2024vernacopter}, or directly solve PDDL planning tasks under few-shot prompting~\cite{silver2024generalized}.

% current llm planenr's limitation
However, base models without post-training exhibit critical deficiencies: they show low planning success rates and cannot guarantee safety. Without domain-specific safety knowledge and alignment with safety-critical decision preferences, LLMs may produce plans that are semantically incorrect, action-infeasible, or violate safety constraints, potentially leading to hazardous behaviors in real-world deployments.

These limitations raise a fundamental question: how can we systematically align LLMs to generate verifiably safe task plans with strong \emph{safety generalization} across domains?

To address these challenges, we propose \emph{SafeGen-LLM} (Safety-Generalizable LLM), a post-training framework that, a post-training framework that enables Large Language Models to perform safety-aware task planning in robotic systems by incorporating verifiable safety knowledge into the training process. As illustrated in Figure~\ref{fig:safe-gen-llm}, the framework consists of three key components:
\begin{enumerate}
    \item \textbf{Safety-aware planning dataset construction}: We introduce a planning dataset that incorporates explicit safety constraints, enabling systematic training and evaluation of safety-aware planning models.
    \item \textbf{Stage I: Supervised Fine-Tuning (SFT)}: Built upon the proposed dataset, we apply SFT to enable the model to learn planning grammar and semantics.
    \item \textbf{Stage II: Group Relative Policy Optimization (GRPO) with fine-grained reward machines}:  We devise a verifiable and fine-grained reward machine to guide GRPO ~\cite{shao2024deepseekmath} training, encouraging the LLM to achieve planning goals while maintaining safety alignment.
\end{enumerate}
To ensure stable training, we adopt curriculum learning by progressively introducing planning problems with increasing complexity, improving both training stability and effectiveness.

\begin{figure*}[htbp!]
  \centering
  \includegraphics[width=0.9\textwidth]{diagram.pdf}
  \caption{Overview of the proposed SafeGen-LLM framework. Stage I performs SFT on formally verified, safety-constrained plans. Stage II applies GRPO using fine-grained reward signals derived from formal verification to enforce safety alignment.}
  \label{fig:safe-gen-llm}
\end{figure*}

Our main contributions are summarized as follows:
\begin{itemize}
    \item \textbf{A unified benchmark for safety-aware PDDL planning.}
    We introduce a dataset benchmark covering multiple robotics-inspired task-planning domains with explicitly defined  safety constraints, enabling systematic training and evaluation of safety compliance and generalization in planning tasks.

    \item \textbf{A systematic post-training framework for safe planning LLMs.}
    We propose a systematic two-stage post-training framework combining SFT and GRPO with fine-grained reward machines derived from formal verification. This approach improves the safety generalization of LLM-based planners, leading to higher planning success rates and more consistent safety compliance across multiple domains.

    \item \textbf{Cross-domain safety generalization with superior performance.}
    Through extensive experiments, we demonstrate that our trained models achieve strong planning performance across multiple domains with diverse safety constraints, effectively solving new problems for each domain while \textbf{outperforming frontier models with orders of magnitude more parameters} in safety-aware planning.
\end{itemize}

% The remainder of this paper is organized as follows.
% Section~\ref{sec:related} reviews related work on classical task planning, AI-based planners, and LLM-based planning.
% Section~\ref{sec:preliminaries} introduces preliminaries on classical planning and PDDL, and formulates the safety-aware and safety-generalizable planning problems.
% Section~\ref{IV} presents the proposed SafeGen-LLM framework.
% Section~\ref{V} describes the experimental setup and results.
% Finally, Section~\ref{VI} concludes with a discussion of limitations and future research directions.

\section{Related Work}\label{sec:related}

\noindent\textbf{Classical task planning.}
Classical planners such as Fast Downward~\cite{helmert2006fast}, Metric-FF~\cite{hoffmann2001ff}, and OPTIC~\cite{benton2012temporal} solve symbolic planning problems by heuristic search over PDDL models. Extensions including HTN planners~\cite{nau2005applications} improve scalability through structural decomposition. However, these methods rely on expert-designed heuristics, scale poorly with problem size, and become brittle under additional safety constraints~\cite{geffner2013concise}.

\noindent\textbf{Learning-based planners.}
Learning-driven approaches include learned heuristics~\cite{yoon2008learning,shivashankar2015hierarchical} and RL-based task policies. However, RL planners suffer from state-space explosion, extensive data requirements, and limited generalization~\cite{shivadekar2025artificial}. Neuro-symbolic approaches~\cite{acharya2023neurosymbolic,lyu2022towards} integrate symbolic reasoning with learned components but remain domain-specific. Overall, existing learning-based planners do not jointly achieve strong generalization and verifiable safety compliance.

\noindent\textbf{LLM-based planning.}
LLMs have shown promise as planners for robotic and task-planning settings~\cite{song2023llm,silver2024generalized}. Methods such as iterative self-refinement~\cite{zhou2024isr} and hybrid LLM-search approaches~\cite{zhang2025lamma} improve plan feasibility, while recent benchmarks reveal that fine-grained constraints expose LLM robustness limitations~\cite{huang2025language}. However, most approaches rely on frozen or lightly adapted models without intrinsically learning safety-aware planning behavior or systematically generalizing safety constraints across domains.


\section{Preliminaries}\label{sec:preliminaries}
This section briefly reviews classical planning and PDDL with safety constraints,
and then introduces the formal robotic task-planning model studied in this work.

\subsection{Classical Planning and PDDL}
In classical planning, a problem is defined as
$\Pi=\langle \mathcal{F}, \mathcal{A}, I, \mathcal{G} \rangle$,
where $\mathcal{F}$ denotes the set of fluents, $\mathcal{A}$ the action set,
$I\subseteq\mathcal{F}$ the initial state, and $\mathcal{G}$ the goal condition.
Each action $a\in\mathcal{A}$ is associated with preconditions $\mathrm{Pre}(a)$
and effects $\mathrm{Eff}(a)$.
A plan is a finite action sequence whose execution from $I$ leads to a state
satisfying $\mathcal{G}$.

Planning problems are commonly specified using
PDDL~\cite{aeronautiques1998pddl}, which provides a structured format
to describe actions, predicates, objects, initial states, and goals.
PDDL has become the de facto standard for representing classical planning benchmarks.

PDDL3 extends PDDL with constructs for specifying temporal constraints and preferences
over plans~\cite{gerevini2005plan}.
In particular, safety constraints can be encoded using the \texttt{:constraints}
field as temporal formulas that restrict all valid plans.
For example, the following constraint enforces that no block is ever placed on top
of a fragile block:
\begin{lstlisting}[style=lispstyle]
(:constraints
  (always
    (forall (?x - fragile-block ?y - block)
      (not (on ?y ?x)))))
\end{lstlisting}
In this work, we focus on hard safety constraints that must be satisfied by all
valid plans.

\subsection{Robotic Task Planning}
We formalize a robotic task-planning problem as a tuple
$\mathcal{P} = (\mathcal{S}, \mathcal{A}, s_0, \mathcal{G}, \mathcal{C})$,
where $\mathcal{S}$ denotes the (possibly hybrid) state space of the robotic system,
$\mathcal{A}$ is the set of available actions or control decisions,
$s_0 \in \mathcal{S}$ is the initial state,
$\mathcal{G}$ is the goal condition,
and $\mathcal{C}$ is a set of formal safety constraints that define the safe operating
region of the system.
Each constraint $C \in \mathcal{C}$ may specify either a state invariant that must hold
at every state along execution or a trajectory-level requirement capturing temporal
patterns such as ``eventually,'' ``until,'' or ``before.''

\paragraph{Safety-Aware Planning.}
A plan is defined as a finite sequence of actions
$\pi = [a_1, a_2, \dots, a_n]$
that induces a state trajectory
$s_0 \xrightarrow{a_1} s_1 \xrightarrow{a_2} \cdots \xrightarrow{a_n} s_n$.
The plan is \emph{goal-reaching} if the final state satisfies the goal condition,
i.e., $s_n \models \mathcal{G}$, and it is \emph{safe} if it respects all safety
constraints, denoted by $\pi \models \mathcal{C}$.
Safety-aware planning requires finding a plan that satisfies both conditions:
\begin{equation}
  \text{find } \pi \text{ s.t. } s_n \models \mathcal{G} \;\land\; \pi \models \mathcal{C}.
\end{equation}
While classical search-based planners enforce safety constraints explicitly during planning, LLM-based planners generate plans as sequences and do not inherently simulate state transitions against formal constraints. As a result, safety must be learned from data or feedback, or enforced through external
verification and correction mechanisms.

\paragraph{Safety-Generalizable Planning.}
Beyond solving individual problems, we require planners to generalize safety reasoning to unseen problems and domains, a property we call \emph{safety generalizability}. We distinguish two forms.

\noindent\textbf{Cross-Problem Safety Generalizability.}
Within a fixed domain $D=(\mathcal{S}, \mathcal{A}, \mathcal{C})$, the model must produce safe and goal-reaching plans for unseen problems with different initial states and goals.

\noindent\textbf{Cross-Domain Safety Generalizability.}
This property further requires safety generalization across a collection of domains $\{D_1, \dots, D_K\}$, each with distinct action semantics and safety constraints. Formally, a model $\mathcal{M}$ is safety-generalizable if:
\begin{equation}
  \forall D_j,\;
  \forall \mathcal{P}_i \in D_j:\;
  \pi_i = \mathcal{M}(\mathcal{P}_i)
  \Rightarrow
  (s_n^i \models \mathcal{G}_i \;\land\; \pi_i \models \mathcal{C}_j).
\end{equation}
\begin{table}[t]
  \centering
  \caption{Domains and their safety constraints.}
  \label{tab:domain-introduction}
  \begin{tabularx}{\linewidth}{l|X|X}
    \hline
    \textbf{Domain} & \textbf{Description} & \textbf{Safety Constraints} \\
    \hline
    \textbf{Blocksworld} &
    Rearrange stacked blocks into a target configuration using pick-and-place actions. &
    Maintain stable stacking order; a supporting block must be placed before the blocks above it. \\
    \hline
    \textbf{Ferry} &
    Transport cars between locations using a ferry. &
    Avoid overloading; ensure cars reach destinations before departure or return. \\
    \hline
    \textbf{Grippers} &
    Use two grippers to move objects to a target room. &
    Prevent simultaneous operation in narrow rooms; restrict certain grippers for safety-critical objects. \\
    \hline
    \textbf{Spanner} &
    Collect spanners and tighten nuts across locations. &
    Tighten bolts in a safe order; limit shared tool access to avoid congestion. \\
    \hline
  \end{tabularx}
\end{table}
\section{Safe-Gen LLM}\label{IV}

In this section, we present \emph{Safe-Gen LLM}, a safety-oriented framework for adapting LLMs to CPS task planning under formal constraints.
The framework consists of two core components.
First, we perform SFT on a curated dataset of safety-compliant plans, injecting domain knowledge and enforcing strict output formats.
Second, we apply Direct Preference Optimization (DPO) to align the model with safety-critical preferences, explicitly distinguishing safe plans from unsafe or infeasible ones.

\subsection{Supervised Fine-Tuning for Safety-Compliant Planning}\label{subsec:sft}
Building on the SFT framework introduced in Section~\ref{sec:preliminaries}, we apply supervised fine-tuning to adapt LLMs for CPS task planning.
In our framework, SFT serves to:
(i) encode domain and safety knowledge,
(ii) enforce syntactic and semantic plan validity, and
(iii) provide a stable reference policy for subsequent preference-based alignment.
The SFT process consists of three steps:
designing domain-specific safety knowledge, constructing the supervised dataset, and fine-tuning the model.

\begin{table*}[h]
  \centering
  \caption{Brief introduction of the domains and their safety constraints.}
  \label{tab:domain-introduction}
  \begin{tabularx}{0.95\linewidth}{l|X|X}
    \hline
    \textbf{Domain} & \textbf{Description} & \textbf{Safety Constraints} \\
    \hline
    \textbf{Blocksworld} &
    The agent must rearrange stacked blocks into a desired configuration using pick-and-place actions. &
    Blocks must be stacked in a safe and stable order, e.g., a supporting block must be positioned before the blocks it supports. \\
    \hline
    \textbf{Ferry} &
    The agent controls a ferry that transports cars between locations. &
    The ferry must never be overloaded, and cars must safely reach their destinations before the ferry departs or returns. \\
    \hline
    \textbf{Grippers} &
    A robot with two grippers must pick up and deliver all objects to a target room. &
    Robots (or manipulators) must avoid operating simultaneously in the same narrow room, and certain grippers may be restricted for safety-critical objects. \\
    \hline
    \textbf{Spanner} &
    A worker must collect spanners and tighten all nuts across connected locations. &
    Bolts must be tightened in a safe sequence (e.g., foundation nuts before upper ones), and access to shared tools (e.g., entering the tool shed) is limited to prevent unsafe congestion. \\
    \hline
  \end{tabularx}
\end{table*}

\noindent\textbf{Domain-specific safety knowledge design.}
We start by selecting five task-planning domains from the open-source PDDL2 problem generators~\cite{seipp-et-al-zenodo2022}.
The domains are chosen using the following criteria:
\begin{itemize}
  \item relevance to real-world CPS task planning;
  \item presence of safety-critical objects, locations, or actions;
  \item availability of problem instances with varying difficulty.
\end{itemize}
Based on these criteria, we select four representative domains: Blocksworld, Ferry, Grippers, and Spanner.
A brief overview of each domain and its associated safety constraints is given in Table~\ref{tab:domain-introduction}.
For each domain, we design additional domain-specific safety knowledge in the form of high-level constraints that mirror realistic CPS requirements (e.g., collision avoidance, load limits, safe ordering of operations).
These constraints are subsequently encoded in PDDL3 \texttt{:constraints} format and used to generate safety-compliant plan demonstrations.

\subsubsection{Dataset construction.}
The pipeline for constructing the SFT dataset is illustrated in Figure~\ref{fig:dataset-construction}.

\begin{figure}[h]
  \centering
  \includegraphics[width=1\linewidth]{sft_diagram.pdf}
  \caption{Pipeline for supervised fine-tuning dataset construction.}
  \label{fig:dataset-construction}
\end{figure}

First, we generate planning problems for each domain using the PDDL2 problem generators.
We remove isomorphic or trivially equivalent problems to reduce redundancy and employ a classical planner to ensure that the remaining problems are feasible.

Second, we encode the domain-specific safety constraints from Table~\ref{tab:domain-introduction} into PDDL3 \texttt{:constraints}.
We then use the temporal PDDL3 planner OPTIC~\cite{benton2012temporal} to solve the resulting constrained planning problems, and verify each candidate solution using the VAL tool~\cite{howey2004val}.
Only solutions that are successfully validated by VAL (i.e., respect both domain preconditions and encoded safety constraints) are retained.

Third, we convert each verified solution into an instruction--response pair.
The instruction consists of a natural-language prompt that includes the planning domain, problem instance, and (when applicable) safety constraints; the response is the corresponding validated plan.
Formally, we obtain a supervised dataset
\[
\mathcal{D}_{\text{SFT}} = \{(x_i, y_i)\}_{i=1}^{N},
\]
where $x_i$ is an input prompt and $y_i$ is the desired output plan.

To enhance data diversity and reduce overfitting, we design ten instruction templates.
For each problem, we randomly sample a fixed number of templates from this pool, which encourages variation in surface forms and prevents the model from simply memorizing a single prompt pattern.
A representative template is shown below.

\begin{userbox}
  You are a planning expert. Your task is to generate a \textbf{valid plan} for the given domain and problem.
  
  \texttt{DOMAIN:}
  \{\{domain\_content\}\}
  
  \texttt{PROBLEM:}
  \{\{problem\_content\}\}
  
  \textbf{Output Requirements:}
  \begin{itemize}
    \item Return \textbf{ONLY} the plan steps, one per line.
    \item Each line must follow the format: \texttt{(<ACTION\_NAME> <param1> <param2> ...)}.
    \item Use only objects defined in the \texttt{PROBLEM}.
    \item Do \textbf{NOT} include any explanations, comments, or headers.
    \item Do \textbf{NOT} output anything except the plan lines.
    \item The output must \textbf{NOT} contain natural language sentences.
    \item If the \texttt{PROBLEM} includes constraints, the plan must satisfy all of them; otherwise, solve as a standard goal-directed task.
    \item Ensure that all action preconditions hold and no constraints or invariants are violated at any step.
  \end{itemize}
  
  \textbf{Plan:}
\end{userbox}

This template explicitly instructs the model to behave as a planning expert and to produce strictly formatted action sequences.
By providing the domain and problem specifications in a structured form and enforcing strong output requirements (no extra text, strict syntax, constraint satisfaction), the template helps ensure that the generated plans remain syntactically correct, executable, and consistent with the safety constraints.

\subsubsection{Supervised fine-tuning.}
Given the constructed dataset $\mathcal{D}_{\text{SFT}}$, we fine-tune the LLM using the standard SFT objective (Eq.~\ref{eq:sft_loss}).
By minimizing this objective, SFT encourages the model to reproduce planning behaviors that are syntactically valid, executable, and safety-compliant.
Beyond learning \emph{what} plan to output, SFT also enforces \emph{how} to output it, such as adhering to strict action syntax and avoiding natural-language explanations.
Moreover, the SFT model serves as a reference policy $\pi_{\text{ref}}$ for subsequent preference-based alignment via DPO.

\subsection{Safety Preference Alignment via DPO}
Building on DPO (Section~\ref{sec:preliminaries}), we apply preference optimization to encode safety-critical preferences on top of the SFT model, biasing the policy toward plans that are not only feasible but also strictly aligned with formal safety constraints.
We decompose this process into two stages: preference data design and model alignment.

\subsubsection{Preference data design.}
We first construct a preference dataset for each domain by generating and verifying candidate plans.
For each plan produced by the LLM planner, we invoke VAL~\cite{howey2004val} to automatically check its behavior against the corresponding PDDL domain and problem (including PDDL3 \texttt{:constraints}).
The verifier outputs are parsed to assess validity, executability, safety, and goal satisfaction.

Based on these verification results, we categorize LLM-generated plans into five error and success types:
\begin{itemize}
  \item \textbf{Plan Format Error}: the plan is syntactically invalid or cannot be parsed and thus cannot be executed or evaluated.
  \item \textbf{Precondition Violation}: one or more action preconditions fail during execution.
  \item \textbf{Safety Constraint Violation}: the plan reaches the goal but violates at least one safety constraint along the execution trace.
  \item \textbf{Goal Not Satisfied}: the plan executes safely but fails to achieve the goal.
  \item \textbf{Success Plan}: the plan both satisfies all safety constraints and achieves the goal.
\end{itemize}

We assume that these categories follow an increasing order of preference, with \textbf{Success Plan} being strictly preferred over all other types, and plans that violate safety constraints being ranked particularly low.
We assign relative preference scores accordingly and construct preference pairs $(y^{+}, y^{-})$ where $y^{+}$ belongs to a higher-ranked category than $y^{-}$.
To ensure diversity and coverage, we sample candidate plans from the SFT model under multiple temperature settings, which induces variation in plan length, structure, and error types.

To further emphasize safety, we additionally incorporate solutions from PDDL2-based problem solvers that \emph{ignore} our added safety constraints.
These plans are feasible with respect to the original domain but violate the new safety constraints, and are therefore labeled as \textbf{Safety Constraint Violation} examples.
By contrasting such unsafe-but-feasible plans with fully safe plans in the preference data, we explicitly teach the model to prefer safety-compliant behavior over merely goal-reaching behavior.

\subsubsection{Model alignment.}
Using the DPO objective (Eq.~\ref{eq:dpo_loss}), we optimize the policy with the SFT model as the reference policy $\pi_{\text{ref}}$.
By constructing preference pairs that explicitly contrast safe and unsafe plans (e.g., success vs.\ safety-violating plans, or safety-violating vs.\ precondition-violating plans), we guide the model toward a policy that favors safety-compliant behaviors while preserving the task competence acquired during SFT.
The resulting Safe-Gen LLM is therefore both instruction-following and safety-aligned.

\subsection{Online Reinforcement Learning via GRPO}

Building on GRPO (Section~\ref{sec:preliminaries}), we apply online reinforcement learning to further improve the model's safety-aware planning capabilities.
Unlike DPO which relies on offline preference data, GRPO allows the model to explore and learn from its own generated plans using verifiable rewards.

\subsubsection{Reward design.}
We design a hierarchical reward function based on the VAL verifier output. For each generated plan, we assign rewards according to a four-stage validation pipeline:
\begin{itemize}
  \item \textbf{Format correctness}: The plan must be syntactically valid and parseable.
  \item \textbf{Precondition satisfaction}: All action preconditions must hold during execution.
  \item \textbf{Safety constraint compliance}: The plan must satisfy all PDDL3 constraints.
  \item \textbf{Goal achievement}: The plan must reach the specified goal state.
\end{itemize}
Plans that pass all stages receive the highest reward, while failures at earlier stages receive progressively lower rewards.

\subsubsection{Curriculum learning.}
To improve training stability, we adopt a curriculum learning strategy that progressively increases problem difficulty. Problems are grouped by complexity (e.g., number of objects, constraint types), and the model is first trained on simpler instances before advancing to harder ones. This approach helps the model build foundational planning skills before tackling complex safety constraints.

The GRPO objective optimizes the policy by comparing multiple sampled responses within each group, encouraging the model to consistently generate high-reward plans while maintaining diversity in its outputs.

\section{Experiments}\label{V}
In this section, we empirically evaluate the proposed framework along four dimensions:
(i)~its runtime behavior compared to a classical search-based planner,
(ii)~its cross-problem safety generalization within the same domain,
(iii)~its cross-domain safety generalization to unseen domains, and
(iv)~its effectiveness in a real-world CPS deployment on a physical robot arm.

\subsection{Experimental Setup}\label{subsec:experimental-setup}
We first describe the datasets, models, and training environment used in our experiments.

\noindent\textbf{SFT dataset construction.}
As introduced in Section~\ref{subsec:sft}, we select five domains from the PDDL2 problem generators~\cite{seipp-et-al-zenodo2022} to construct the safety-oriented SFT dataset: \emph{Blocksworld}, \emph{Ferry}, \emph{Grippers}, \emph{Spanner}, and \emph{Delivery}.
We use Blocksworld, Ferry, Grippers, and Spanner as training domains and reserve Delivery as a held-out test domain.
For each domain, we generate $500$ planning problems with parameters ranging from simple to complex to diversify the dataset (e.g., different numbers of blocks, objects, or locations).
All problems are solved by the temporal planner OPTIC~\cite{benton2012temporal} and the resulting plans are validated by VAL~\cite{howey2004val} to ensure correctness.
Importantly, these problems cannot be solved by the solutions of the corresponding PDDL2 benchmark problems without incorporating the additional safety constraints, which highlights the need for a safety-aware planner.

For each problem instance, we randomly sample $5$ instruction templates from a pool of $10$ templates to construct natural-language prompts, thereby further diversifying the dataset.
This procedure yields an SFT dataset of $10{,}000$ instruction--response pairs.

\noindent\textbf{DPO dataset construction.}
For the DPO stage, we construct a preference dataset that distinguishes safety-compliant plans from unsafe or infeasible ones.
First, we use the fine-tuned SFT model to sample candidate plans under two temperature settings, $T=0.6$ and $T=0.9$, which encourages diverse error types in the generated plans.
Second, we treat valid plans from the SFT dataset---i.e., plans that both achieve the goal and satisfy all safety constraints---as the \emph{chosen} candidates.
Third, we collect \emph{rejected} candidates from two sources:
(i) plans sampled from the LLM that fail to reach the goal or violate safety constraints, and
(ii) solutions to the corresponding PDDL2 problems that are feasible in the original domain but violate the additional safety constraints; these are categorized as \textbf{Safety Constraint Violation} examples.
Overall, we construct a DPO preference dataset containing $11{,}960$ instruction--response pairs.
For both SFT and DPO, the test set consists of $50$ problems per domain.

\noindent\textbf{Model selection and training environment.}
We instantiate our framework with open-source LLMs available through the Unsloth library~\cite{unsloth} on Hugging Face~\cite{wolf-etal-2020-transformers}.
Specifically, we consider:
\begin{itemize}
  \item \texttt{unsloth/mistral-7b-instruct-v0.3-bnb-4bit}, abbreviated as \textbf{Mistral-7B};
  \item \texttt{unsloth/Qwen3-14B-unsloth-bnb-4bit}, abbreviated as \textbf{Qwen3-14B}.
\end{itemize}
Unless otherwise specified, we use the quantized 4-bit variants of these models throughout our experiments.

For training, Mistral-7B is fine-tuned on a single NVIDIA A10 GPU with 24~GB memory, and Qwen3-14B is fine-tuned on a single NVIDIA H100 GPU with 80~GB memory.
Our implementation is based on Python~3.10 and PyTorch~2.8.0~\cite{paszke2019pytorch}, and uses the Hugging Face Transformers library~\cite{wolf-etal-2020-transformers}, TRL~\cite{vonwerra2022trl}, and Unsloth~\cite{unsloth}.
We adopt LoRA~\cite{hu2022lora} and QLoRA~\cite{dettmers2023qlora} as parameter-efficient fine-tuning strategies to reduce memory usage and training time.

\subsection{Running Time Comparison}
We first compare the runtime behavior of an LLM-based solver with a classical search-based planner to evaluate scalability as problem complexity grows.

We generate planning problems whose parameter sizes range from $3$ to $50$.
Taking Blocksworld as an example, the simplest instance contains $3$ blocks, whereas the most complex instance contains $50$ blocks to be stacked.
For each parameter size, we generate $20$ problem instances, resulting in $96$ problems per domain.
We consider two domains: Blocksworld and Grippers.
As the LLM-based solver, we use GPT-OSS-20B~\cite{openai2025gptoss120bgptoss20bmodel}, and as the classical baseline, we use OPTIC~\cite{benton2012temporal}.
GPT-OSS-20B runs on a single NVIDIA A100 GPU with 80~GB memory, while OPTIC runs on an AMD EPYC~9554 64-core CPU with 180~GB memory.
The timeout for both methods is set to $100$ seconds.

Figure~\ref{fig:running-time-comparison} reports the average solving time across different problem sizes.

\begin{figure}[h]
  \centering
  \begin{subfigure}[b]{0.48\linewidth}
    \centering
    \includegraphics[width=\linewidth]{figures/blocksworld_time.png}
    \caption{Blocksworld domain.}
    \label{fig:blocksworld-time}
  \end{subfigure}
  \hfill
  \begin{subfigure}[b]{0.48\linewidth}
      \centering    
    \includegraphics[width=\linewidth]{figures/grippers_time.png}
    \caption{Grippers domain.}
    \label{fig:grippers-time}
  \end{subfigure}
  \caption{Running time comparison for the LLM-based solver and classical solver in two domains.}
  \label{fig:running-time-comparison}
\end{figure}

We observe from Figure~\ref{fig:running-time-comparison} that, for relatively simple problems, the classical search-based solver is much faster than the LLM-based solver and typically completes within one second.
However, as the problem complexity increases, the runtime of the classical solver grows dramatically---almost exponentially in these settings---and eventually reaches the timeout limit of $100$ seconds.
In contrast, the runtime of the LLM-based solver remains relatively stable across different problem sizes.

Because LLM decoding is inherently stochastic, the runtime of the LLM-based solver may occasionally decrease slightly as the problem size increases within a certain range; nevertheless, the overall trend is consistent.
These results suggest that the LLM-based solver can effectively address the scalability limitations of classical search-based planners in complex planning domains.

\begin{figure}[h]
  \centering
  \includegraphics[width=0.9\linewidth]{figures/blocksworld_model_comparison.png}
  \caption{SFT and DPO results comparison for Mistral-7B in the Blocksworld domain.}
  \label{fig:model-comparison}
\end{figure}

\subsection{Cross-Problem Safety Generalizability}
We next evaluate the cross-problem safety generalization of the proposed framework, i.e., whether a model trained on a set of problems and safety constraints within a given domain can solve previously unseen problems with new safety constraints in the \emph{same} planning domain.

We use the Blocksworld domain as a representative example and fine-tune Mistral-7B.
We generate $500$ training problems with different numbers of blocks (from $3$ to $6$) and varying operator combinations using the PDDL2 problem generators.
During the SFT stage, we train the model for three epochs with a batch size of $4$, resulting in $714$ gradient steps.
The training loss converges steadily to $0.0262$.

For the subsequent DPO stage, following the construction procedure described in Section~\ref{subsec:experimental-setup}, we obtain a preference dataset with $3{,}050$ instruction--response pairs in Blocksworld.
Figure~\ref{fig:model-comparison} compares the performance of the pretrained, SFT, and DPO models on the Blocksworld test set.

We summarize the key observations as follows:
\begin{enumerate}
  \item \textbf{Reduction in precondition violations.}
  The precondition violation rate is significantly reduced after SFT and DPO. 
  In particular, the rate decreases from nearly $98\%$ for the pretrained model to $52\%$ for the SFT model and $42\%$ for the DPO model.
  This indicates that the SFT and DPO models have internalized the underlying planning domain knowledge and are able to generate feasible plans that respect domain preconditions.

  \item \textbf{Improvement in safety constraint satisfaction.}
  The rate of safety constraint violations also decreases after DPO, from $28\%$ for the SFT model to $26\%$ for the DPO model.
  This demonstrates that the DPO stage effectively leverages preference data to further bias the model toward safety-compliant plans.

  \item \textbf{Improvement in overall success rate.}
  The success rate---defined as generating a plan that both satisfies all safety constraints and reaches the goal---is $0\%$ for the pretrained model, but increases to $14\%$ and $20\%$ for the SFT and DPO models, respectively.
\end{enumerate}

It is worth noting that we observe two test cases categorized as \textbf{Plan Format Error}.
This is because such error types were not included in the preference dataset; they could be further reduced by adding a small number of format-error examples into the preference data.
We also emphasize that all experiments in this section are conducted with a quantized $7$B LLM and a relatively small dataset (only $500$ problems and roughly three thousand instruction--response pairs).
Therefore, the reported results likely underestimate the potential performance of the framework, yet they are still encouraging and demonstrate its effectiveness.

\begin{figure}[!ht]
  \centering
  \includegraphics[width=0.9\linewidth]{figures/blocksworld_cross_domain_comparison.png}
  \caption{Blocksworld model performance on the Delivery domain.}
  \label{fig:blocksworld_delivery_comparison}
\end{figure}

\begin{table*}[h]
  \centering
  \caption{Error Type Percentages by Domain}
  \label{tab:error_percentages}
  \resizebox{0.95\textwidth}{!}{%
  \begin{tabular}{l|cc|cc|cc|cc||cc}
  \hline
  \textbf{Domain} 
    & \multicolumn{2}{c}{Plan Format Error} 
    & \multicolumn{2}{c}{Precondition Violation} 
    & \multicolumn{2}{c}{Safety Constraint Violation} 
    & \multicolumn{2}{c}{Goal Not Satisfied} 
    & \multicolumn{2}{c}{Success Plans} \\
  \hline
    
     & \textbf{Pretrained} & \textbf{SFT} 
     & \textbf{Pretrained} & \textbf{SFT} 
     & \textbf{Pretrained} & \textbf{SFT} 
     & \textbf{Pretrained} & \textbf{SFT} 
     & \textbf{Pretrained} & \textbf{SFT} \\
  \hline
  \textbf{Blocksworld} & 0.0\% & 0.0\% & 70.0\% & 58.0\% & 24.0\% & 24.0\% & 2.0\% & 6.0\% & 4.0\% & \textbf{12.0\%} \\
  \textbf{Ferry      } & 0.0\% & 2.0\% & 98.0\% & 44.0\% & 2.0\% & 36.0\% & 0.0\% & 0.0\% & 0.0\% & \textbf{18.0\%} \\
  \textbf{Grippers   } & 8.0\% & 6.0\% & 42.0\% & 42.0\% & 46.0\% & 30.0\% & 4.0\% & 8.0\% & 0.0\% & \textbf{14.0\%} \\
  \textbf{Spanner    } & 6.0\% & 2.0\% & 94.0\% & 48.0\% & 0.0\% & 0.0\% & 0.0\% & 0.0\% & 0.0\% & \textbf{50.0\%} \\
  \hline
    \multicolumn{11}{c}{\textbf{Generalized Domain}} \\
  \hline
  \textbf{Delivery   } & 0.0\% & 0.0\% & 44.0\% & 22.0\% & 48.0\% & 64.0\% & 2.0\% & 2.0\% & 6.0\% & \textbf{12.0\%} \\
  \hline
  \end{tabular}
  }% end resizebox
  \end{table*}

\subsection{Cross-Domain Safety Generalizability}
We now consider cross-domain safety generalization, which is the ultimate goal of the proposed framework: a trained LLM-based planner should be able to solve problems in \emph{unseen} domains with \emph{unseen} safety constraints.

\subsubsection{Blocksworld Model Performance on an Unseen Domain.}
We first evaluate the Blocksworld-only model on the unseen Delivery domain.
The results are shown in Figure~\ref{fig:blocksworld_delivery_comparison}.
We observe that both the SFT- and DPO-trained Blocksworld models perform poorly on the Delivery domain: the success rate is essentially zero, and most plans fail due to precondition violations.
The performance is comparable to that of the pretrained Mistral-7B model.

This result indicates that a model trained solely on Blocksworld does not acquire sufficient cross-domain safety generalization to solve problems in the structurally different Delivery domain, highlighting a key challenge for fine-tuning-based LLM planners.
\begin{table*}[!ht]
  \centering
  \caption{Error Type Percentages by Domain on SFT and DPO Models}
  \label{tab:error_percentages_sft_dpo}
  \resizebox{0.95\textwidth}{!}{%
  \begin{tabular}{l|cc|cc|cc|cc||cc}
  \hline
  \textbf{Domain}
  & \multicolumn{2}{c}{Plan Format Error}
  & \multicolumn{2}{c}{Precondition Violation}
  & \multicolumn{2}{c}{Safety Constraints Violation}
  & \multicolumn{2}{c}{Goal Not Satisfied}
  & \multicolumn{2}{c}{Success Plans} \\
  \hline
  
  & \textbf{SFT} & \textbf{DPO}
  & \textbf{SFT} & \textbf{DPO}
  & \textbf{SFT} & \textbf{DPO}
  & \textbf{SFT} & \textbf{DPO}
  & \textbf{SFT} & \textbf{DPO} \\
  \hline
  \textbf{Blocksworld} & 0.0\% & 2.0\% & 58.0\% & 40.0\% & 24.0\% & 36.0\% & 6.0\% & 6.0\% & 12.0\% & \textbf{16.0\%} \\
  \textbf{Ferry      } & 2.0\% & 0.0\% & 44.0\% & 44.0\% & 36.0\% & 26.0\% & 0.0\% & 6.0\% & 18.0\% & \textbf{24.0\%} \\
  \textbf{Grippers   } & 6.0\% & 6.0\% & 42.0\% & 20.0\% & 30.0\% & 46.0\% & 8.0\% & 4.0\% & 14.0\% & \textbf{24.0\%} \\
  \textbf{Spanner    } & 2.0\% & 16.0\% & 48.0\% & 26.0\% & 0.0\% & 2.0\% & 0.0\% & 0.0\% & 50.0\% & \textbf{56.0\%} \\
  \hline
  \multicolumn{11}{c}{\textbf{Generalized Domain}} \\
  \hline
  \textbf{Delivery   } & 0.0\% & 0.0\% & 22.0\% & 28.0\% & 64.0\% & 46.0\% & 2.0\% & 2.0\% & 12.0\% & \textbf{24.0\%} \\
  \hline
  \end{tabular}
  }% end resizebox
\end{table*}
\subsubsection{Pretrained and SFT Model Performance Comparison.}
To address this limitation, we next train the model jointly on four domains: Blocksworld, Ferry, Grippers, and Spanner.
The SFT stage uses the full SFT dataset of $10{,}000$ instruction--response pairs, and the DPO stage uses the $11{,}960$ preference pairs described earlier.

Table~\ref{tab:error_percentages} reports the error-type distributions for the pretrained and SFT models across the four training domains and the unseen Delivery domain.
We highlight several key findings:
\begin{enumerate}
\item \textbf{Success rate improvements across domains.}
In Blocksworld, the success rate improves from $4\%$ (pretrained) to $12\%$ (SFT).
For Ferry, Grippers, and Spanner, the pretrained model achieves $0\%$ success, whereas the SFT model reaches $18\%$, $14\%$, and $50\%$, respectively.
These results indicate that the SFT model exhibits strong cross-problem safety generalization, enabling it to generate valid plans for previously unseen problem instances within the four training domains.

\item \textbf{Significant reduction in precondition violations.}
Across the four training domains, the precondition violation rate is substantially reduced after SFT:
\begin{itemize}
  \item Ferry: from $98\%$ to $44\%$,
  \item Blocksworld: from $70\%$ to $58\%$,
  \item Grippers: remains at $42\%$,
  \item Spanner: from $94\%$ to $48\%$.
\end{itemize}
This demonstrates that the SFT model effectively internalizes domain-level planning rules and produces plans that are more frequently executable with respect to preconditions.

\item \textbf{Interpreting safety constraint violations.}
In some domains, the Safety Constraint Violation rate is higher for the SFT model than for the pretrained model.
This behavior is expected: the pretrained model often fails to generate meaningful or executable plans, so its outputs may not progress to states where safety constraints are even evaluated.
By contrast, the SFT model generates more complete and structured plans, thereby exposing more opportunities for safety checks---and hence observable safety violations---to occur.

\item \textbf{Generalization to the unseen Delivery domain.}
Most importantly, the SFT model also improves performance on the unseen Delivery domain.
The precondition violation rate decreases from $44\%$ to $22\%$, while the success rate increases from $6\%$ to $12\%$.
This indicates that SFT injects planning knowledge that can transfer across domains, enabling the model to generate feasible plans even in a domain that was not seen during supervised training.
\end{enumerate}

Overall, these results show that SFT substantially enhances the model's ability to generate feasible, legal, and safety-aware plans across both seen and unseen domains.



\subsubsection{SFT and DPO Model Performance Comparison.}
We finally study the impact of DPO on safety generalization.
Starting from the SFT model trained on the four domains, we further fine-tune it using the preference dataset via DPO.
Table~\ref{tab:error_percentages_sft_dpo} reports the error-type distributions for the SFT and DPO models.

We obtain the following observations:
\begin{enumerate}
    \item \textbf{Success rate improvements after DPO.}  
    DPO consistently improves success rates across all domains:
    from $12\%$ to $16\%$ in Blocksworld, from $18\%$ to $24\%$ in Ferry, from $14\%$ to $24\%$ in Grippers, and from $50\%$ to $56\%$ in Spanner.
    This indicates that DPO effectively leverages preference data to bias the model toward plans that are both safe and goal-satisfying.

    \item \textbf{Cross-domain safety generalization.}  
    In the unseen Delivery domain, DPO reduces the Safety Constraint Violation rate from $64\%$ to $46\%$, a substantial improvement in safety behavior.
    Combined with the success rate increase from $12\%$ to $24\%$, this shows that DPO internalizes safety-related patterns that generalize beyond the training domains and transfer to new, unseen domains.
\end{enumerate}

Overall, the relatively small preference dataset and the use of quantized models limit the absolute performance achievable by DPO.
Nevertheless, the two-stage SFT+DPO pipeline yields a clear improvement over the pretrained and SFT-only baselines, enabling the model to generate safe and goal-satisfying plans with both cross-problem and cross-domain safety generalization.

\begin{figure*}[!ht]
  \centering
  \includegraphics[width=0.9\linewidth]{Blocksworld.pdf}
  \caption{Case study in the Blocksworld domain.}
  \label{fig:case-study}
\end{figure*}

\subsection{Real-World Validation in Blocksworld}

To demonstrate the practical impact of safety-aware training, we evaluate our framework in both simulation and a physical Cyber-Physical System (CPS) setup using the classical Blocksworld domain.

\paragraph{Simulation.}
We first select a test-set problem instance and compare two planners:
(i) a classical PDDL2 solver, and
(ii) our safety-aware SFT+DPO model.
The task requires satisfying a safety constraint that enforces a specific ordering between two stacking operations to avoid unsafe intermediate configurations.

As shown in Figure~\ref{fig:case-study}, the classical solver produces a plan that violates the constraint, leading to an intermediate configuration that would cause a collision between blocks.  
In contrast, the safety-aware LLM planner restructures the action sequence so that the constraint is satisfied while still achieving the goal, demonstrating that the learned safety knowledge directly influences plan generation.

\paragraph{Physical Deployment.}
We further deploy the safety-aware planner on a desktop robot arm (Elephant myCobot~280) controlled by a Raspberry~Pi.  
The LLM-generated plan is transmitted via SSH and executed on the real robot.  
We compare the safety-aware plan with a baseline unsafe sequence that mirrors the violation observed in simulation.

Figure~\ref{fig:experiment-snapshot} shows snapshots from the execution.  
The unsafe baseline results in a physical collision during stacking, whereas the safety-aware plan completes the task without violating the safety requirement.  
This experiment confirms that the proposed framework not only improves symbolic safety in simulation but also yields robust, collision-free behaviors when deployed on real CPS hardware.


\begin{figure}[h]
  \centering
  \includegraphics[width=0.9\linewidth]{experiments.pdf}
  \caption{Snapshot of the physical robot executing the Blocksworld task.}
  \label{fig:experiment-snapshot}
\end{figure}

\section{Conclusion}\label{VI} In this paper, we proposed a formal safety–guided fine-tuning framework that enables Large Language Models to perform safety-aware task planning for Cyber-Physical Systems. By combining Supervised Fine-Tuning with Direct Preference Optimization on a safety-constrained planning dataset, our approach injects verifiable safety knowledge into LLMs and aligns their behavior with safety-critical preferences. Extensive experiments on five PDDL-based domains demonstrate three key findings. First, the fine-tuned models substantially improve success rates while reducing precondition violations, indicating that they have effectively acquired domain-level planning knowledge. Second, the learned planners exhibit cross-problem and cross-domain safety generalization: they can solve unseen problems with new safety constraints in the training domains and achieve non-trivial performance in an unseen domain. Third, the LLM-based solver shows favorable scalability compared to a classical search-based planner, maintaining relatively stable runtime as problem complexity increases. Our study is conducted with quantized lightweight models and a moderately sized safety dataset, suggesting that the reported performance is still conservative. Future work includes scaling the framework to more complex CPS settings, integrating richer formal verification tools, and exploring automatic construction or refinement of safety constraints from interaction data.



\bibliographystyle{ACM-Reference-Format}
\bibliography{sample-base}
\end{document}
